\chapter{Introdução}
\label{chap:intro} %este label será usado para referenciar este capítulo

%As primeiras frases têm a missão de prender a atenção do leitor e por isso são as mais importantes do texto. Diga o quanto antes o que você fez e quais são os resultados alcançados. Ao terminar de ler a introdução o leitor tomará uma nova decisão de se vale a pena ou não continuar lendo o texto. Capte a atenção do leitor bem aqui.

%A comunicação escrita é considerada umas das cinco habilidades mais importantes por profissionais de engenharia e um engenheiro passa em média mais de $25$\% do seu tempo escrevendo \cite{eggert2002response,spretnak1982survey}. Uma quantidade similar de tempo é gasta na escrita de correspondência e de relatórios técnicos \cite{cunningham2012perceptions}. Dessa forma, encare a escrita do seu projeto como um treinamento nessa importante habilidade.

%Neste texto você encontrará não apenas uma estrutura para escrever seu trabalho em \LaTeX, mas também um pequeno manual de boas práticas na escrita técnica. Leia com atenção e coloque as sugestões em prática à medida que preenche o texto com o conteúdo do seu próprio projeto. Também será apresentado um número de vícios de escrita comumente encontrados nas monografias de alunos. 

%A seguir está a estrutura de organização sugerida pelo colegiado do curso. Note que ela não é necessariamente a melhor para contar a história do seu projeto. Você pode por exemplo preferir usar títulos mais pertinentes ao seu contexto. Contudo, o seu texto deve conter cada um dos pontos a seguir.

\section{Motivação e Justificativa}
\label{sec:motivacao}

O avanço contínuo na área de controle e automação exige que os futuros engenheiros dominem técnicas práticas e teóricas para solucionar problemas reais de engenharia. No contexto do curso de Engenharia de Controle e Automação, é essencial que os alunos tenham acesso a plataformas de aprendizagem que unam teoria e prática de maneira dinâmica e interativa. Nesse cenário, o desenvolvimento de sistemas experimentais versáteis e acessíveis é uma abordagem eficiente para consolidar o aprendizado e despertar o interesse por sistemas de controle complexos.

O projeto proposto consiste na construção de um robô que se equilibra sobre uma esfera, uma aplicação concreta do conceito de pêndulo invertido. Esse sistema é amplamente utilizado no ensino de técnicas de controle devido à sua natureza desafiadora e suas características dinâmicas, que exigem uma compreensão aprofundada de modelagem, análise e implementação de controladores.

A utilização de modelagem 3D e impressão em impressoras 3D para a construção do robô garante um custo reduzido, acessível tanto para o laboratório quanto para alunos interessados em replicar o projeto. Essa abordagem também assegura que eventuais danos à planta possam ser facilmente reparados, incentivando o uso contínuo e livre do sistema pelos discentes. Dessa forma, o projeto atende a dois objetivos principais: a oferta de um sistema dinâmico e interessante para o estudo de controle e a promoção de um ambiente de aprendizagem prático e sustentável.

%Argumente sobre a importância do projeto desenvolvido usando uma visão de alto nível, sem entrar em detalhes. Contextualize seu projeto dentro do local de execução ou da literatura e explique como ele é necessário ou inovador. É possível fazer uma breve revisão bibliográfica, confrontando seu trabalho com outras referências bibliográficas para mostrar a sua contribuição. No quesito contribuição, é muito importante deixar claro o tempo todo que partes do projetos foram executadas por outros e que partes foram executadas por você. Caso contrário, corre-se o risco de inadvertidademente tomar crédito pelo trabalho de outrem, o que pode ter implicações legais. 

\section{Objetivos do Projeto}
\label{sec:objetivos}

Tendo em vista o exposto acima, este projeto tem por objetivos:

\begin{enumerate}[a)]
\item  Projetar e construir um robô do tipo pêndulo invertido que se equilibra sobre uma esfera, utilizando conceitos aprendidos ao longo do curso de Engenharia de Controle e Automação;
\item Tornar a construção viável e replicável por meio da utilização de impressão 3D, proporcionando durabilidade e acessibilidade; 
\item Implementar e validar técnicas de controle avançadas para garantir a estabilidade do sistema e seu funcionamento eficiente.

Esses objetivos foram estabelecidos com a finalidade de integrar o conhecimento teórico adquirido ao longo do curso com aplicações práticas relevantes, contribuindo para a formação de engenheiros mais preparados para lidar com desafios reais.
\end{enumerate}

%O conteúdo desta seção pode se sobrepor um pouco com o da seção anterior, podendo ela ser um sumário dos pontos expostos anteriormente. A escolha do título da seção talvez seja mais apropriada para a fase de proposta do projeto. Afinal, nesta fase se conhecem os objetivos e não os resultados. Por outro lado, fará pouco sentido discutir objetivos quando o projeto está finalizado, especialmente se tais objetivos não foram alcançados. 


\section{Local de Realização}
\label{sec:empresa}

O desenvolvimento do projeto será realizado no Laboratório de Controle da Escola de Engenharia da Universidade Federal de Minas Gerais (UFMG). A impressão das peças será viabilizada por meio de uma impressora 3D disponibilizada pelo professor Fernando de Oliveira Souza, que também supervisionará as etapas práticas e fornecerá o suporte técnico necessário.


\section{Estrutura da Monografia}
\label{sec:organizacao}

Esta monografia está organizada em quatro capítulos, além desta introdução. O Capítulo 1 apresenta uma visão geral do projeto, incluindo sua motivação, objetivos e local de execução. O Capítulo 2 explora os princípios fundamentais relacionados ao pêndulo invertido, detalhando conceitos essenciais para o entendimento do sistema e do controle aplicado. O Capítulo 3 descreve a metodologia adotada no projeto, abrangendo as etapas de modelagem 3D, construção da planta física, modelagem matemática e projeto dos controladores. Por fim, o Capítulo 4 apresenta as conclusões obtidas, as contribuições do projeto e sugestões para trabalhos futuros, além de relatar os principais desafios enfrentados.


\clearpage